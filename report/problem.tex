
\section{Постановка задачи} \label{sec:1-postanovka-zadachi}
Приведем систему уравнений, описывающую нестационарное двумерное движение вязкого баротропного газа:
\begin{equation}
\left\{
	\begin{aligned}
    	&\cfrac{\partial \rho}{\partial t} + \cfrac{\partial \rho u_1}{\partial x_1}+ \cfrac{\partial \rho u_2}{\partial x_2} = 0, \\
	    & 
\cfrac{\partial \rho u_1}{\partial t} + \cfrac{\partial \rho u_1^2}{\partial x_1} + \cfrac{\partial \rho u_1 u_2}{\partial x_2} 
	+ \cfrac{\partial p}{\partial x_1} 
	= \mu \Big(\cfrac{4}{3}\cfrac{\partial^2 u_1}{\partial x_1^2} + \cfrac{\partial^2 u_1}{\partial x_2^2}
	+  \cfrac{1}{3}\cfrac{\partial^2 u_2}{\partial x_1\partial x_2} \Big) + \rho f_1, \\
		&
\cfrac{\partial \rho u_2}{\partial t} + \cfrac{\partial \rho u_1 u_2}{\partial x_1} + \cfrac{\partial \rho u_2^2}{\partial x_2}
	+ \cfrac{\partial p}{\partial x_2} 
	= \mu \Big(\cfrac{1}{3}\cfrac{\partial^2 u_1}{\partial x_1\partial x_2} + \cfrac{\partial^2 u_2}{\partial x_1^2}
	+  \cfrac{4}{3}\cfrac{\partial^2 u_2}{\partial^2 x_2} \Big) + \rho f_2, \\
    	& p = p (\rho).
	\end{aligned}
\right.
\label{eq:1-nachal-zadacha}
\end{equation}
Здесь $\mu$ -- коэффициент вязкости газа (известная неотрицательная константа), $p$ -- давление газа (известная функция), $f$ -- вектор внешних сил (известная функция).

Неизвестные функции $\rho$ и $u$, плотность и скорость соответственно, -- функции от двух переменных $t$ и $x$ (переменные Эйлера), причем $$(t, x) \in Q = [0, T] \times \Omega.$$

В качестве граничных условий берется следующее:
\begin{equation}
\rho|_{\Gamma^-} = \rho_{\gamma} = 1,\quad u_1|_{\Gamma^-} = \omega \in \{0,1;\  1\},\quad \cfrac{\partial u_1}{\partial x_1}\Big|_{\Gamma^+} = 0. \label{eq:1-usl2}
\end{equation}

На оставшейся границе компоненты скорости равны нулю, а функция плотности считается неизвестной.

Для решения задачи введем равномерную сетку $\omega_h$ с шагом $h_x$ по оси $x$, с шагом $h_y$ по оси $y$ и с шагом $\tau$ по оси $t$. 
Введем константы $M_x$, $M_y$ и $N$, такие что $X = Mh$, $Y= My$ и $T = N\tau$.
